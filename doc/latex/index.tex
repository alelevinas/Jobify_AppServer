CI\+: \href{https://travis-ci.org/alelevinas/Jobify_AppServer}{\tt } Coverage\+: \href{https://coveralls.io/github/alelevinas/Jobify_AppServer?branch=master}{\tt }

Application Server para el proyecto de Taller de Programación II

Instalación\+:

Las únicas librería que usamos y no compilamos directamente son leveldb y restclient-\/cpp, por lo que es necesario ejecutar los comandos

\subsection*{6. Instalación}

Hay 3 formas\+:
\begin{DoxyItemize}
\item Compilando el código fuente
\item Instalando el paquete {\ttfamily .deb} en sistemas debian
\item Utilizando el {\ttfamily Dockerfile}
\end{DoxyItemize}

\subparagraph*{Chats}

Primero hace falta descargar las librerías necesarias y luego el código fuente. Una vez descargados, hace falta ejecutar {\ttfamily cmake} para luego compilar con {\ttfamily make Jobify\+Appserver} un script {\ttfamily yaml} de ejemplo se ve acontinuación\+: 
\begin{DoxyCode}
1 install:
2   - sudo pip install cpp-coveralls
3   - sudo apt-get install libsnappy-dev libleveldb-dev
4   #para jwtcpp
5   - sudo apt-get install libcrypto++9-dbg libcrypto++-dev
6   - sudo apt-get install libjansson4 libjansson-dev libjansson-dbg
7   - sudo apt-get install autoconf libtool
8   - git clone https://github.com/mrtazz/restclient-cpp.git
9   - cd restclient-cpp
10   - ./autogen.sh
11   - ./configure
12   - sudo make install
13   - export LD\_LIBRARY\_PATH=/usr/local/lib:$LD\_LIBRARY\_PATH
14   - cd ..
15 
16 script:
17   - mkdir build
18   - cd build
19   - cmake ..
20   - make JobifyAppserver
21   - ./Jobify\_Appserver
\end{DoxyCode}


\subparagraph*{Paquete {\ttfamily .deb}}

Una vez descargado el paquete debe ejecutar el siguiente comando\+: {\ttfamily sudo dpkg -\/i Jobify\+\_\+\+App\+Server-\/0.\+1.\+1-\/\+Linux.\+deb}

El ejecutable se instalará en {\ttfamily /home/usuario/local/bin} por lo que estará en el {\ttfamily P\+A\+TH} y se podrá ejecutar en cualquier terminal. Las bases de datos se almacenarán en {\ttfamily /home/usuario/.Jobify\+\_\+\+App\+Server/bds}

\subparagraph*{Dockerfile}

Simplemente tener descargado Docker \href{https://docs.docker.com/engine/installation/linux/ubuntulinux/}{\tt install} Luego ubicarse en el mismo directorio del {\ttfamily Dockerfile} y ejecutar el comando {\ttfamily docker build .}

Documentación A\+PI\+: \href{http://rebilly.github.io/ReDoc/?url=https://raw.githubusercontent.com/alelevinas/Jobify_AppServer/master/jobify-appserver-serverAPI.yaml}{\tt link} 