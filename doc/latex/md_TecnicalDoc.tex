\#\+Documentacion Técnica 

 {\itshape Proyecto\+:} Jobify App\+Server {\itshape Author\+:} Grupo 3 {\itshape Created\+:} 30/11/2016 {\itshape Version\+:} 1.\+0.\+0 



\subsection*{Table of Contents}


\begin{DoxyEnumerate}
\item Servidor H\+T\+ML
\item Bases de datos
\item J\+S\+ON
\item L\+OG
\item Protocolo de acceso
\item Instalación 


\end{DoxyEnumerate}

\subsection*{1. Servidor H\+T\+TP}

Para montar el servidor se utilizó la libreria mongoose-\/cpp. \href{https://github.com/Gregwar/mongoose-cpp}{\tt Github}. La misma consiste en una encapsulación de mongoose de forma tal que para manejar los distintos endpoints se utilizan controladores. Los controladores son los que registran las rutas, el H\+T\+TP verb y el handler que actuará cuando un cliente le pega. Actualmente la aplicación consta de 3 controladores\+: \paragraph*{\hyperlink{classProfileController}{Profile\+Controller}}

Se utiliza para todo lo relacionado con alta, baja, modificación y actualización de perfiles de usuario. También para los servicios de agregar contacto, recomendar y búsquedas con filtros. \paragraph*{\hyperlink{classChatController}{Chat\+Controller}}

Se utiliza para el manejo de chats. Enviar, pedir una conversación o borrar un mensaje entre otras cosas. \paragraph*{\hyperlink{classSharedServerController}{Shared\+Server\+Controller}}

Se utiliza para entregar información antes consultada ante el Shared\+Server. Son skills, job\+\_\+positions y categories.



 \subsection*{2. Bases de datos}

Para el almacenado de la información se utilizó la librería leveldb. La misma es no\+S\+QL y almacena cualquier tipo de dato en una relación {\itshape clave-\/valor} Para manejarse con Json creamos la clase \hyperlink{classDB}{DB} con métodos que reciben valores en J\+S\+ON y parsean antes de devolver. Hay 4 bases de datos principales\+:


\begin{DoxyItemize}
\item Users
\item Accounts
\item Sessions
\item Chats
\end{DoxyItemize}

\subparagraph*{Users}

Esta base de datos almacena todo el perfil del usuario en formato json. Un ejemplo de usuario es\+: 
\begin{DoxyCode}
1 \{
2         "chats" : ["id1","id2"],
3         "contacts" : ["nuan@a.com", "nico@b.com"],
4         "coordenates" : "",
5         "dob" : "19/08/1993",
6         "email" : "ale.l@hotmail.com",
7         "firebase\_token" :
       "ftpNLANtYrg:APA91bG1SMVr9u3DSsKV88XVxDjKnAzix3ZLHr1jiKvOIq1rF2LM9PZffhvAjkEAJxzxdWo\_Tlq-oX2xSO0P5iQC30QnsxXjydYh4NwY3l5TRIcS8iiFUn51vhiAgRY6Zt82MKnXOH-V",
8         "gender" : "male",
9         "name" : "Alejandro Levinas",
10         "picture" : "",
11         "recommended\_by" : ["leo@a.com", "pedro@a.com"],
12         "username" : "ale.l@hotmail.com"
13 \}
\end{DoxyCode}
 \subparagraph*{Accounts}

La base de datos almacena usuarios y contraseñas. \subparagraph*{Sessions}

Almacena información de la sesión actual. Esto es, username y timestamp. La clave es el Token, que es una combinación de username, password y timestamp encryptados. Es el mismo token que se devuelve al cliente para acceder a la A\+PI. \subparagraph*{Chats}

Almacena los chats. Las claves son {\itshape username1\+\_\+username2} o {\itshape username2\+\_\+username1}. Contiene un array de mensajes en el que cada mensaje contiene el {\itshape username} del emisor, el {\itshape timestamp} y el cuerpo del mensaje. 

 \subsection*{3. J\+S\+ON}

Para el manejo de J\+S\+ON se utilizó la librería {\itshape jsoncpp} \href{https://github.com/open-source-parsers/jsoncpp}{\tt Github}



 \subsection*{4. L\+OG}

El log se realizó mediante la librería {\itshape easylogging++$\ast$ \href{https://github.com/easylogging/easyloggingpp}{\tt Github} 

 \subsection*{5. Protocolo de acceso}}

{\itshape  Para acceder a las funcionalidades principales del servidor hace falta estar autentificado. Al hacer $\ast$/signup}, un cliente da de alta un usuario con un {\itshape username} y una {\itshape password}. La respuesta a ese endpoint es un {\itshape token}. Este {\itshape token} tiene una duración y solo es válido mientras no expire. Para la utilización de la mayoriá de los endpoint hace falta incluir el {\itshape token} en los headres H\+T\+TP. Es un header con clave {\ttfamily Token} y valor el {\itshape token} recibido. Ej {\ttfamily Token\+: 55\+F\+C\+F21\+B40\+F8\+A9\+A1\+D56\+D95\+F302\+D\+A9045}

\subsection*{6. Instalación}

Hay 3 formas\+:
\begin{DoxyItemize}
\item Compilando el código fuente
\item Instalando el paquete {\ttfamily .deb} en sistemas debian
\item Utilizando el {\ttfamily Dockerfile}
\end{DoxyItemize}

\subparagraph*{Chats}

Primero hace falta descargar las librerías necesarias y luego el código fuente. Una vez descargados, hace falta ejecutar {\ttfamily cmake} para luego compilar con {\ttfamily make Jobify\+Appserver} un script {\ttfamily yaml} de ejemplo se ve acontinuación\+: 
\begin{DoxyCode}
1 install:
2   - sudo pip install cpp-coveralls
3   - sudo apt-get install libsnappy-dev libleveldb-dev
4   #para jwtcpp
5   - sudo apt-get install libcrypto++9-dbg libcrypto++-dev
6   - sudo apt-get install libjansson4 libjansson-dev libjansson-dbg
7   - sudo apt-get install autoconf libtool
8   - git clone https://github.com/mrtazz/restclient-cpp.git
9   - cd restclient-cpp
10   - ./autogen.sh
11   - ./configure
12   - sudo make install
13   - export LD\_LIBRARY\_PATH=/usr/local/lib:$LD\_LIBRARY\_PATH
14   - cd ..
15 
16 script:
17   - mkdir build
18   - cd build
19   - cmake ..
20   - make JobifyAppserver
21   - ./JobifyAppserver
\end{DoxyCode}


\subparagraph*{Paquete {\ttfamily .deb}}

Una vez descargado el paquete debe ejecutar el siguiente comando\+: {\ttfamily sudo dpkg -\/i Jobify\+\_\+\+App\+Server-\/0.\+1.\+1-\/\+Linux.\+deb}

El ejecutable se instalará en {\ttfamily /home/usuario/local/bin} por lo que estará en el {\ttfamily P\+A\+TH} y se podrá ejecutar en cualquier terminal. Las bases de datos se almacenarán en {\ttfamily /home/usuario/.Jobify\+\_\+\+App\+Server/bds}

\subparagraph*{Dockerfile}

Simplemente tener descargado Docker \href{https://docs.docker.com/engine/installation/linux/ubuntulinux/}{\tt install} Luego ubicarse en el mismo directorio del {\ttfamily Dockerfile} y ejecutar el comando {\ttfamily docker build .} 